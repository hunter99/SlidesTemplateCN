\documentclass[CJK,notheorems,compress,mathserif,table]{beamer}
\usepackage{ifxetex}
\usepackage{ifpdf}

%\useoutertheme[height=0.1\textwidth,width=0.15\textwidth,hideothersubsections]{sidebar}
%\usecolortheme{whale}      % Outer color themes: whale, seahorse, dolphin
%\usecolortheme{orchid}     % Inner color themes: lily, orchid
%\useinnertheme[shadow]{rounded}
%\setbeamercolor{sidebar}{bg=blue!50}
%\setbeamercolor{background canvas}{bg=blue!9}
%\usefonttheme{serif}
\setbeamertemplate{navigation symbols}{}
\usetheme{default}

%%------------------------常用宏包---------------------------------------------------------------------

\mode<article> % 仅应用于article版本
{
  \usepackage{beamerbasearticle}
  \usepackage{fullpage}
  \usepackage{hyperref}
}
%\usepackage[font=Times, timeinterval=10, timeduration=30,
%timewarningfirst=85, timewarningsecond=90,
%fillcolorwarningsecond=white!60!yellow]{tdclock}
%\setbeameroption{show notes}
\setbeameroption{hide notes}
%\usepackage{ifpdf}
%% 下面的包控制beamer的风格,可以根据自己的爱好修改
\usepackage{beamerthemesplit}   % 使用split风格
\usepackage{beamerthemeshadow}  % 使用shadow风格
\usepackage{amsmath,amssymb}
\ifxetex
  \usepackage[BoldFont,SlantFont,CJKchecksingle]{xeCJK}
  \setCJKmainfont[BoldFont=SimHei]{SimSun}
  \setCJKmonofont{SimSun}% 设置缺省中文字体

%  \DeclareGraphicsExtensions{.pdf,.jpg,.jpeg,.png}
\else
  \usepackage{CJKutf8}
  \usepackage{CJKnumb}
  \usepackage{CJKpunct}
  \usepackage{CJKspace}
%	\ifpdf
%	  \usepackage[pdftex]{graphicx}
%	  \graphicspath{./pics/}
%	  \DeclareGraphicsExtensions{.pdf,.jpeg,.png}
%	\else
%	  \usepackage[dvips]{graphicx}
%	  \graphicspath{{./pics/}}
%	  \DeclareGraphicsExtensions{.eps,.ps}
%	\fi
\fi
\usepackage{graphicx}
\graphicspath{{./pics/}}

%\usepackage{CJKutf8}
\usepackage{hyperref}
\hypersetup{CJKbookmarks=true}
%\usepackage{graphicx}
%\graphicspath{{./pics/}}
%%\DeclareGraphicsRule{*}{mps}{*}{}
\usepackage{subfigure}
\usepackage{xmpmulti}
\usepackage{colortbl,dcolumn}
\usepackage{multimedia}
\usepackage{chemarr}
%pgf由beamer的开发者tantau开发,由以在beamer中精确作图。
\usepackage{pgf,pgfarrows,pgfnodes,pgfautomata,pgfheaps}

%% 下面的代码用来读入Logo图象
%\pgfdeclaremask{logomask}{pics/thu-logo1-mask.eps}
%\pgfdeclareimage[mask=logomask,height=1.5cm]{logo}{pics/thu-logo1.eps}

%设置logo图标
%\logo{\includegraphics[height=0.09\textwidth]{pics/thu-logo2.jpg}}
\logo{\includegraphics[height=0.09\textwidth,bb=0 0 489 490]{pics/thu-logo2.png}}

\renewcommand{\raggedright}{\leftskip=0pt \rightskip=0pt plus 0cm}
\raggedright

\def\hilite<#1>{%
\temporal<#1>{\color{blue!35}}{\color{magenta}}%
{\color{blue!75}}}

\newcolumntype{H}{>{\columncolor{blue!20}}c!{\vrule}}
\newcolumntype{H}{>{\columncolor{blue!20}}c}
%==================================参考文献==============================================================
\newcommand{\upcite}[1]{\textsuperscript{\cite{#1}}}  %自定义命令\upcite, 使参考文献引用以上标出现
%\bibliographystyle{plain}% set in slidereference.tex
%%%%%%%%%%%%%%%%%%%%%%%%%%%%%%%%%%%%%%重定义字体、字号命令 %%%%%%%%%%%%%%%%%%%%%%%%%%%%%%%%%%%%%%%%%%%%%%
\newcommand{\songti}{\CJKfamily{song}}        % 宋体
\newcommand{\fangsong}{\CJKfamily{fs}}        % 仿宋体
\newcommand{\kaishu}{\CJKfamily{kai}}         % 楷体
\newcommand{\heiti}{\CJKfamily{hei}}          % 黑体
\newcommand{\lishu}{\CJKfamily{li}}           % 隶书
\newcommand{\youyuang}{\CJKfamily{you}}       % 幼圆
\newcommand{\sihao}{\fontsize{14pt}{\baselineskip}\selectfont}      % 字号设置
\newcommand{\xiaosihao}{\fontsize{12pt}{\baselineskip}\selectfont}  % 字号设置
\newcommand{\wuhao}{\fontsize{10.5pt}{\baselineskip}\selectfont}    % 字号设置
\newcommand{\xiaowuhao}{\fontsize{9pt}{\baselineskip}\selectfont}   % 字号设置
\newcommand{\liuhao}{\fontsize{7.875pt}{\baselineskip}\selectfont}  % 字号设置
\newcommand{\qihao}{\fontsize{5.25pt}{\baselineskip}\selectfont}    % 字号设置
%%%%%%%%%%%%%%%%%%%%%%%%%%%%%%%%%%%%%%%%%%%%%%%%%%%%%%%%%%%%%%%%%%%%%%%%%%%%%%%%%%%%%%%%%%%%%%%%%%%%%%%%
\newcommand{\argmax}{\operatornamewithlimits{argmax}}
\newcommand{\argmin}{\operatornamewithlimits{argmin}}
\newcommand{\tfn}[1]{\footnote{\tiny #1}}
\begin{document}
%\initclock
\ifxetex
\else
\begin{CJK}{UTF8}{song}
\fi
\ifxetex
%\pgfdeclaremask{thumask}{pics/thu-mark1-mask.png}
%\pgfdeclareimage[mask=thumask,height=0.25cm]{thu}{pics/thu-mark1.png}
\pgfdeclareimage[height=0.25cm]{thu}{pics/thu-mark1.png}
\pgfdeclareimage[height=0.25cm]{thunew}{pics/thu-mark1.png}

%\pgfdeclaremask{titlelogomask}{pics/thu-gate2-mask.jpg}
%\pgfdeclareimage[mask=titlelogomask,height=2.5cm]{titlelogo}{pics/thu-gate2.jpg}
\pgfdeclareimage[height=2.5cm]{titlelogo}{pics/thu-gate2.jpg}
\pgfdeclareimage[height=2.5cm]{titlelogonew}{pics/thu-gate2.jpg}
\else
\ifpdf
\pgfdeclaremask{thumask}{pics/thu-mark1-mask.png}
\pgfdeclareimage[mask=thumask,height=0.25cm]{thu}{pics/thu-mark1.png}
\pgfdeclareimage[height=0.4cm]{thunew}{pics/thu-mark1.png}

\pgfdeclaremask{titlelogomask}{pics/thu-gate2-mask.jpg}
\pgfdeclareimage[mask=titlelogomask,height=2.5cm]{titlelogo}{pics/thu-gate2.jpg}
\pgfdeclareimage[height=2.5cm]{titlelogonew}{pics/thu-gate2.jpg}
\else
\pgfdeclaremask{thumask}{pics/thu-mark1-mask.eps}
\pgfdeclareimage[mask=thumask,height=0.25cm]{thu}{pics/thu-mark1.eps}

\pgfdeclaremask{titlelogomask}{pics/thu-gate2-mask.eps}
\pgfdeclareimage[mask=titlelogomask,height=2.5cm]{titlelogo}{pics/thu-gate2.eps}
\fi
\fi

%  \begin{CJK*}{UTF8}{song}
%%----------------------- Theorems ---------------------------------------------------------------------
%\theoremstyle{plain} %\theoremheaderfont{\heiti}
%\theorembodyfont{\songti} \theoremindent0em
%\theoremseparator{\hspace{1em}} \theoremnumbering{arabic}
%\theoremsymbol{} %定理结束时自动添加的标志
%\newtheorem{problem}{Problem. }
\newtheorem{problem}{问题}
\newtheorem{asm}{假定}
\newtheorem{theorem}{定理}
%\newtheorem{theorem}{Theorem}
\newtheorem{definition}{定义}
%\newtheorem{definition}{Definition.}
\newtheorem{lemma}{引理}
\newtheorem{corollary}{推论}
%\newtheorem{corollary}{Corollary}
\newtheorem{proposition}{命题}
\newtheorem{example}{例}
\newtheorem{remark}{注}
\newtheorem{solution}{解答}
\newcommand\term[1]{#1}
\def\semichecked{\checkmark\!\!\!\raisebox{0.4 em}{\tiny$\smallsetminus$}}
\definecolor{Gray}{gray}{0.75}
%\renewcommand\figurename{\rm 图}
%\renewcommand\tablename{\bf 表}
\renewcommand\figurename{\rm Figure}
\renewcommand\tablename{\bf Table}

\AtBeginSection[]{ % 在每个Subsection前都会加入的Frame
  \frame<handout:0>{
    \frametitle{目录}
%\tableofcontents[current,hideallsubsections]%
\tableofcontents[sectionstyle=show/shaded,subsectionstyle=show/show/hide]%
%    \tableofcontents[currentsubsection]
%\tableofcontents[current,currentsubsection]%currentsection(或 currentsubsection) 仅正常显示当前节(小节)目录,其他部 分半透明 
  }
}
\AtBeginSubsection[]{ % 在每个Subsection前都会加入的Frame
  \frame<handout:0>{
    \frametitle{目录}
%\tableofcontents[current,hideothersubsections,currentsubsection]%
	\tableofcontents[sectionstyle=show/shaded,subsectionstyle=show/shaded/hide]
%\tableofcontents[current,currentsubsection]%
%    \tableofcontents[currentsubsection]
  }
}

\setbeamertemplate{background canvas}[vertical shading][bottom=white,top=structure.fg!25]
\ifxetex
%\titlegraphic{\pgfuseimage{titlelogonew}}
\else
\ifpdf
%\titlegraphic{\pgfuseimage{titlelogonew}}
\else
%\titlegraphic{\pgfuseimage{titlelogo}}
\fi
\fi
\makeatletter
\newcommand{\setfoot}[1]{
\usefoottemplate{ %重新定义页脚,加入作者,单位,单位图标,和文档标题
  \vbox{\tiny%
    \hbox{%
      \setbox\beamer@linebox=\hbox to\paperwidth{%
		\ifxetex
		%\hbox to.5\paperwidth{\hfill\tiny\color{white}\textbf{\insertshortauthor\quad\insertshortinstitute}\hskip.1cm\lower 0.35em\hbox{\pgfuseimage{thunew}}\hskip.3cm}%
        \hbox to.5\paperwidth{\hfill\tiny\color{white}\textbf{
		\insertpagenumber/90\quad\insertshortinstitute}\color{black}#1\hskip.0cm\lower 0.35em\hbox{\pgfuseimage{thunew}}\hskip.3cm}%
		\else
	  	\ifpdf
        \hbox to.5\paperwidth{\hfill\tiny\color{white}\textbf{\insertshortauthor\quad\insertshortinstitute}\hskip.1cm\lower 0.35em\hbox{\pgfuseimage{thunew}}\hskip.3cm}%
		\else
        \hbox to.5\paperwidth{\hfill\tiny\color{white}\textbf{\insertshortauthor\quad\insertshortinstitute}\hskip.1cm\lower 0.35em\hbox{\pgfuseimage{thu}}\hskip.3cm}%
		\fi
		\fi
        \hbox to.5\paperwidth{\hskip.3cm\tiny\color{white}\textbf{\insertshorttitle}\hfill}\hfill}%
      \ht\beamer@linebox=2.625ex%
      \dp\beamer@linebox=0pt%
      \setbox\beamer@linebox=\vbox{\box\beamer@linebox\vskip1.125ex}%
      \color{structure}\hskip-\Gm@lmargin\vrule width.5\paperwidth
      height\ht\beamer@linebox\color{structure!70}\vrule width.5\paperwidth
      height\ht\beamer@linebox\hskip-\paperwidth%
      \hbox{\box\beamer@linebox\hfill}\hfill\hskip-\Gm@rmargin}
  }
}
}
\setfoot{}
%\usefoottemplate{ %重新定义页脚,加入作者,单位,单位图标,和文档标题
%  \vbox{\tiny%
%    \hbox{%
%      \setbox\beamer@linebox=\hbox to\paperwidth{%
%		\ifxetex
%		%\hbox to.5\paperwidth{\hfill\tiny\color{white}\textbf{\insertshortauthor\quad\insertshortinstitute}\hskip.1cm\lower 0.35em\hbox{\pgfuseimage{thunew}}\hskip.3cm}%
%        \hbox to.5\paperwidth{\hfill\tiny\color{white}\textbf{\insertpagenumber\quad\insertshortinstitute}\hskip.0cm\lower 0.35em\hbox{\pgfuseimage{thunew}}\hskip.3cm}%
%		\else
%	  	\ifpdf
%        \hbox to.5\paperwidth{\hfill\tiny\color{white}\textbf{\insertshortauthor\quad\insertshortinstitute}\hskip.1cm\lower 0.35em\hbox{\pgfuseimage{thunew}}\hskip.3cm}%
%		\else
%        \hbox to.5\paperwidth{\hfill\tiny\color{white}\textbf{\insertshortauthor\quad\insertshortinstitute}\hskip.1cm\lower 0.35em\hbox{\pgfuseimage{thu}}\hskip.3cm}%
%		\fi
%		\fi
%        \hbox to.5\paperwidth{\hskip.3cm\tiny\color{white}\textbf{\insertshorttitle}\hfill}\hfill}%
%      \ht\beamer@linebox=2.625ex%
%      \dp\beamer@linebox=0pt%
%      \setbox\beamer@linebox=\vbox{\box\beamer@linebox\vskip1.125ex}%
%      \color{structure}\hskip-\Gm@lmargin\vrule width.5\paperwidth
%      height\ht\beamer@linebox\color{structure!70}\vrule width.5\paperwidth
%      height\ht\beamer@linebox\hskip-\paperwidth%
%      \hbox{\box\beamer@linebox\hfill}\hfill\hskip-\Gm@rmargin}
%  }
%}
\makeatother
%
